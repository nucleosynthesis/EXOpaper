\begin{section}{Results}

A simultaneous fit to the signal region across the three event categories, allowing for sytematic uncertainty variations of the background expectations, is performed.
The corresponding comparisons between data and background in the \ETm distributions, for each of the three categories, after this fit are shown in Figure~\ref{fig:post_fit_plots}.   
Agreement between the expected SM backgrounds and data is observed at the percent level across the three categories. The largest single-bin local significance across the three categories, is 1.9$\sigma$ and corresponds to the excess seen in the last \ETm bin of the monojet category.

\begin{figure*}[hbtp]\begin{center}
  \subfloat[][]{
 \includegraphics[width=0.5\textwidth]{figures/plot_config_monojet.pdf}
}
  \subfloat[][]{
 \includegraphics[width=0.5\textwidth]{figures/plot_config_resolved.pdf}
}\\
  \subfloat[][]{
 \includegraphics[width=0.5\textwidth]{figures/plot_config_boosted.pdf}
}
 \caption{
   Post-fit distributions of \ETm expected from SM backgrounds and observed in data in the signal region. The expected distributions are 
   evaluated after fitting to the observed data simultaneously across the monojet (a), resolved (b) and boosted (c) categories. 
   The gray bands indicate the post-fit uncertainty on the background, assuming no signal. The expected distribution assuming vector mediated DM production 
   is shown for a DM mass of 10 GeV and a mediator mass of 1 TeV.
 }
 \label{fig:post_fit_plots}\end{center}\end{figure*}

Exclusion limits are set for these models using the CLs method~\cite{cls} with a profile likelihood ratio as the 
test-statistic in which systematic uncertainties are modelled as nuisance parameters. 
For each signal hypothesis tested, upper limits are placed on the ratio of 
the signal cross-section to the predicted cross-section, denoted as $\mu=\sigma/\sigma_{TH}$. Limits are presented in terms of excluded regions in the 
$m_{\mathrm{MED}}-m_{\textrm{DM}}$ plane, assuming either scalar,
pseudoscalar, vector, and axial-vector mediators, and determining the points for which $\mu\ge1$ is excluded at 90\% CL or more.
Experimental systematic uncertainties, including jet and \ETm response and resolution, are included on the signal model as nuisance parameters, while the
theoretical systematic uncertainties on the inclusive cross-section (20\% and 30\% for the vector and axial-vector, and scalar and pseudoscalar models respectively) due to QCD scale and 
PDF uncertainties are instead added as additional contours to the
exclusion limits. These uncertainties are chosen to be conservative
across the full range in mediator mass from 10~\GeV to 3~\TeV.

To compare direct detection experiments with collider experiments, the direct detection bounds can be interpreted under the Lagrangians given in Equation~\ref{eq:LS}. The limits obtained in 
the simplified models use the standard approaches to compute t-channel scattering~\cite{Kurylov:2003ra,Hisano:2010ct, Cheung:2013pfa,Buchmueller:2014yoa}. 
For the vector and scalar mediator models, the limits are compared with the measurements by LUX~\cite{Akerib:2012ys,Akerib:2013tjd,Szydagis:2014xog} which currently 
provides the strongest constraints for $m_{\rm{DM}} \gtrsim 6$ GeV. For axial-vector couplings, the limits are compared with 
DM-proton-scattering limits from PICO-2L~\cite{Amole:2015lsj}..
For pseudoscalar interactions, direct detection bounds are strongly velocity suppressed. 
The most appropriate comparison is therefore to the most sensitive bounds from indirect detection from FermiLAT~\cite{Ackermann:2011wa,Abdo:2010ex}. 
These limits apply to the scenario in which dark matter is annihilated in the center of a galaxy producing a $\gamma$ ray signature. 
The results are also compared, for all four types of mediators, to constraints obtained from the observed cosmological relic density of DM as determined from 
measurements of the cosmic microwave background by the WMAP and Planck experiments~\cite{Bennett:2003ba,Planck:2006aa}. The expected DM abundance is estimated, separately for each
model, using a thermal freeze-out mechanism implemented in MadDM~\cite{Backovic:2013dpa}, and compared to the observed cold DM density $\Omega_c*h^2=0.12$~\cite{Ade:2013zuv}. 
It is assumed that the simplified model hypothesised provides the only relevant BSM dynamics for DM interactions.

Figure~\ref{fig:masslims} shows the 90\% CL exclusions for the vector, axial-vector, 
scalar and pseudoscalar mediator models.  The 90\% confidence level upper limit on the ratio of excluded cross-section to the predicted cross-section ($\mu_{\textrm{up}}$), 
when assuming the mediator only couples to fermions, is shown by the blue color scale. These limits are calculated under the assumption 
that only the initial state partons and the DM 
particle contribute to the width of the mediator (minimum width constraint)~\cite{An:2012va,Abercrombie:2015wmb,Fox:2011pm,simplified1}. 
%For the vector mediator, the direct-detection bounds dominate above $m_{DM}=6$~\GeV, while for the axial-vector, scalar, pseudoscalar mediator models, the bounds 
%
Under the vector mediator model, the direct detection bounds dominate across most of the plane, while for the axial-vector, there is good complementarity between the direct detection limits and 
those from this analysis. Limits in the scalar mediator scenario are
more sensitive than those from direct detection for small Dark Matter masses. Additional sensitivity is gained
for larger mediator masses in this scenario above 350 \GeV due to the
rise in cross-section of the gluon fusion loop process for masses
greater than twice the
top quark mass. In the pseudoscalar mediator scenario, the limits from this analysis exceed the reach in $m_{\mathrm{MED}}$ byeond those from FermiLAT over the whole region.

\begin{figure}[htbp]
  \centering
  \subfloat[][]{
	\includegraphics[angle=0,width=0.50\textwidth]{figures/MassLimit_1_800_0_Both.pdf}
	\label{fig:mass_800}
  }
  \subfloat[][]{
	\includegraphics[angle=0,width=0.50\textwidth]{figures/MassLimit_1_801_0_Both.pdf}
	\label{fig:mass_801}
  }\\
  \subfloat[][]{
	\includegraphics[angle=0,width=0.50\textwidth]{figures/MassLimit_1_805_0_Both.pdf}
	\label{fig:mass_805}
  }
  \subfloat[][]{
	\includegraphics[angle=0,width=0.50\textwidth]{figures/MassLimit_1_806_0_Both.pdf}
	\label{fig:mass_806}
  }
  \caption{90\% CL Exclusion contours in the $m_{\textrm{med}}-m_{\textrm{DM}}$ plane assuming a vector (a), axial-vector (b), scalar (c), or pseudoscalar (d) mediator. 
The blue scale shows the 90\% CL upper limit on the signal strength assuming the mediator only couples to fermions. For the scalar and pseudoscalar mediators, the exclusion 
contour assuming coupling only to fermions is explicitly shown in the orange line. The white region shows model points which were not tested when assuming coupling only 
to fermions and are not expected to be excluded by this analysis under this assumption.
The excluded region is to the bottom-left of the contours shown in all cases except for that from the relic density as indicated by the shading.
In all of the mediator models, a minimum width is assumed\label{fig:masslims}.}
\end{figure}


%Figures~\ref{fig:xslims_800},~\ref{fig:xslims_801} and~\ref{fig:xslims_805} show the same exclusion contours, this time translated into the 
%plane of $m_{\textrm{DM}}-\sigma_{\textrm{DD}}$, where $\sigma_{\textrm{DM}}$ is 
%the spin-independent/dependant (vector and scalar/axial-vector) DM-nucleon scattering cross-section. 
%These representations allow for a more direct comparison with the direct-detection expedients which typically set 
%model-independent limits on these cross-sections. It should be noted that the limits set from this 
%analysis are however only valid under the simplified model considered, and in particular 
%assuming $g_{DM}=g_{SM}=1$. For the scalar model, it is assumed that only heavy quarks 
%(top and bottom) contribute. Such a choice limits the sensitivity for direct 
%detection, however it allows for direct comparison between collider and direct detection without an additional assumptions 

%on the light quark couplings~\cite{Harris:2015kda}.
Figures~\ref{fig:xslims_800},~\ref{fig:xslims_801} and~\ref{fig:xslims_805} show the same exclusion contours, this time translated into the 
plane of $m_{\textrm{DM}}-\sigma_{\textrm{SI/SD}}$, where $\sigma_{\textrm{SI/SD}}$ are 
the spin-independent/dependent (vector and scalar/axial-vector) DM-nucleon scattering 
cross-sections. These representations allow for a more direct comparison with limits from the direct-detection experiments which typically set 
upper limits on these cross-sections~\cite{ Malik:2014ggr,Harris:2015kda}. It should be noted that the limits set from this 
analysis are however only valid under the simplified model considered, and in particular 
assuming $g_{\textrm{DM}}=g_{\textrm{SM}}=1$. For the scalar model, it is assumed that only heavy quarks 
(top and bottom) contribute. Such a choice limits the sensitivity for direct 
detection, however it allows for direct comparison between collider and direct detection without an additional assumption 
on the light quark couplings~\cite{Harris:2015kda}.  
For the vector and scalar mediator models, direct-detection limits are stronger than 
those obtained in this analysis except in the scenario where the Dark Matter mass is less than around 6 \GeV. For the axial-vector mediator model, the 
limits obtained in this analysis dominate up to around $m_{\mathrm{DM}}=300$ \GeV. 
 
The excess observed in FermiLAT data has led to speculations that Dark Matter annihilation is mediated by a light pseudoscalar~\cite{Calore:2014nla}. 
The production mechanism for these $\gamma$ rays can be interpreted under Dark Matter annihilation to b-quarks allowing for direct 
comparison with limits from this analysis~\cite{Buchmueller:2015eea,Buckley:2014fba,Harris:2014hga}. Figure~\ref{fig:xslims_806} shows 
the exclusion contours assuming pseudoscalar mediation in the plane of DM pair annihilation cross-section versus $m_{\textrm{DM}}$. 
Again, it is assumed that only heavy quarks (top and bottom quarks) contribute in the production of the mediator while for  
the interpretation of the limits in the annihilation cross-section, it is assumed that the mediator only decays to b-quark pairs. 
%Such a choice limits the sensitivity for direct 
%detection, however it allows for direct comparison between collider and direct detection without an additional assumptions 
%on the light quark couplings~\cite{Harris:2015kda}. 
As with all interpretations, the DM particle is assumed to be a Dirac fermion.
The 68\% CL preferred regions in this plane assuming the annihilation of DM pairs to light-quarks (qq), tau or bottom pairs, using data from FermiLAT, 
are shown as solid colour regions. Under the simplified model used, all of these regions are excluded by this analysis.

%\begin{figure}[htbp]
%  \centering
%	\includegraphics[angle=0,width=0.70\textwidth]{figures/MassLimit_1_806_0_Both_DD.pdf}
%  \caption{90\% CL Exclusion contours in the $m_{\textrm{med}}-\sigma_{\textrm{DM}}$ plane assuming a pseudoscalar mediator. 
%The orange line shows the exclusion contours assuming the mediator only couples to fermions. 68\% CL preferred regions, using data from FermiLAT, for DM annihilation 
%to light-quarks (qq), tau pairs ($\tau\tau$) and bottom-quark pairs (bb) are shown by the solid green, pink and brown coloured regions respectively.\label{fig:xslims}}
%\end{figure}

\begin{figure}[htbp]
  \centering
  \subfloat[][]{
	\includegraphics[angle=0,width=0.50\textwidth]{figures/MassLimit_1_800_0_Both_DD.pdf}
	\label{fig:xslims_800}
  }
  \subfloat[][]{
	\includegraphics[angle=0,width=0.50\textwidth]{figures/MassLimit_1_801_0_Both_DD.pdf}
	\label{fig:xslims_801}
  }\\
  \subfloat[][]{
	\includegraphics[angle=0,width=0.50\textwidth]{figures/MassLimit_1_805_0_Both_DD.pdf}
	\label{fig:xslims_805}
  }
  \subfloat[][]{
	\includegraphics[angle=0,width=0.50\textwidth]{figures/MassLimit_1_806_0_Both_DD.pdf}
	\label{fig:xslims_806}
  }
  \caption{90\% CL Exclusion contours in the $m_{\textrm{DM}}-\sigma_{\textrm{DM}}$ plane assuming a vector (a), axial-vector (b), scalar (c), or pseudoscalar (d) mediator. 
For the scalar and pseudoscalar case, the orange line shows the exclusion contours assuming the mediator only couples to fermions. The excluded region in all plots is to the top 
left of the contours shown. In the vector and axial-vector scenarios, limits are shown independently for monojet, boosted and resolved categories. The partial combination of 
the "V-tagged" categories (resolved and boosted) categories is shown for which the boosted category provides the dominant contribution.
In all of the mediator models, a minimum width is assumed. For the pseudoscalar, 68\% CL preferred regions, using data from FermiLAT, for DM annihilation 
to light-quarks (qq), tau pairs ($\tau\tau$) and bottom-quark pairs (bb) are shown by the solid green, pink and brown coloured regions respectively. }
\end{figure}

