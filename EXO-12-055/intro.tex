%%%%%%%%%%%%%%%%%%%%%%%%%%%%%%%%%%%%%%%%%%%%%%
%%%%%%%%%%%%%%%%%%%%%%%%%%%%%%%%%%%%%%%%%%%%%%
\section{Introduction}
%%%%%%%%%%%%%%%%%%%%%%%%%%%%%%%%%%%%%%%%%%%%%%
%%%%%%%%%%%%%%%%%%%%%%%%%%%%%%%%%%%%%%%%%%%%%%
This paper describes a search for Dark Matter (DM) in events containing an energetic narrow or wide jet and an imbalance of transverse momentum (\ETm) in a dataset 
of pp collisions provided by the Large Hadron Collider (LHC) at a centre-of-mass energy of 8 TeV. The data were collected using the Compact Muon Solenoid (CMS) 
detector and correspond to an integrated luminosity of 19.7\fbinv. 

The existence of Dark Matter is one of the most compelling sources of evidence for physics beyond the Standard Model (SM) with a number of astrophysical observations 
suggesting an abundance of a non-baryonic form of matter in the universe. In many theories which extend the SM, the production of DM particles in high-energy hadron-hadron 
collisions, such as those produced at the LHC is realised via a mediator which couples to both the SM and the DM particles.  
The ``monojet'' search provides a model-independent means of exploring DM production at the LHC~\cite{monojet1,monojet2} while 
 ``mono-V'', V=W or Z boson, searches~\cite{monolep,Aad:2014vka,Aad:2013oja,ATLAS:2014wra} target the associated production of DM with SM vector bosons, 
which can be enhanced in theories with non-universal DM couplings~\cite{IVDM}.  The interpretation of results from these and other DM searches at 
the LHC have generally utilized effective field theories (EFT) that assume heavy mediators and DM production via contact interactions~\cite{Fox:2011pm}.  
The results of this analysis are interpreted in the context of a spin-0, and spin-1 mediator decaying to Dark Matter deonted simplified DM models,~\cite{simplified1,Buchmueller:2013dya,Buchmueller:2014yoa}, compatible with the recent recommendation of the ATLAS-CMS LHC DM forum~\cite{Abercrombie:2015wmb}, which span a broad range of 
mediator and DM particle properties. This allows for a comparison in sensitivity with respect to direct detection experiments while retaining validity as a description of DM 
production across the entire kinematic region accessible at the LHC. 

This search is the first at CMS to target the hadronic decay modes of the vector bosons in the mono-V channels. A multivariate V-tagging technique is 
employed to identify the individual jets of moderately boosted vector bosons. The exploration of mono-V production at high boost 
utilises recently developed techniques designed to exploit information 
available in the sub-structure of jets. The events are categorised according to the nature of the jets in the event and the signal extraction is performed by 
considering the $\ETm$ distribution in each event category. These two features provide improved sensitivity compared to the previous CMS monojet analysis~\cite{monojet1}. 


%~\cite{Abercrombie:2015wmb}.

%The canonical ``monojet'' search strategy provides a model-independent means of exploring this scenario~\cite{monojet1,monojet2} while 
%related ``mono-V'' (V=W/Z) searches~\cite{monolep,monoZHbb} target the associated production of DM with SM vector bosons, 
%which can be enhanced in theories with non-universal DM couplings~\cite{IVDM}.  The interpretation of results from these and other DM searches at 
%the Large Hadron Collider (LHC) have generally utilized effective field theories (EFT) that assume heavy mediators and DM production via contact interactions~\cite{Fox:2011pm}.  

This paper is structured as follows; Section 2 outlines the Dark Matter models explored as signal hypotheses and Section 3 provides a 
description of the physics object reconstruction and the event selection and categorisation used in the search. Section 4 describes the 
background modelling used for the signal extraction. The results of this analysis and interpretations of the results in the context of 
simplified models for DM production are presented in Section 5.

