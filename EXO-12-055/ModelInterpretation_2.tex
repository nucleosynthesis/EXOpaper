%\section{Signal Intepretations }
\section{Results}
The signal is split into the four separate lagrangions, scalar,pseudoscalar,vector,axial. An extension to the scalar and pseudo-scalar can be performed by allowing the (pseudo-)scalar interactions to undergo electroweak symmetry breaking (EWSB) in an analogous way to 
the Higgs mechanism~\cite{somepapers}. For (pseudo-)scalar models where EWSB is not present, this is denoted as \emph{fermionic} dark matter production.

In collider experiments, the production of dark matter in spin-0 mediated interactions is predominantly through gluon-fusion via a top-quark loop. 
When EWSB is present in the model, mono-V signatures are produced through the well known higgs-strahlung%???Jargon% 
process. For spin-1 signatures, dark matter is produced in an analogous way to Z boson production through quark initiated processes. The mono-V or monojet signatures follow 
from the presence of a radiated V-boson or jet in the initial state. In all models, the coupling of the mediator to standard model coupling is taken as unity ($g_{SM}=1$) along with 
the dark matter coupling ($g_{DM}=1$). For the (pseudo-)scalar models, this denotes a Yukawa coupling to standard model particles. 
For all models, the width is fixed under the minimum width constraint~\cite{philnote}.

The search is performed through  a simultaneous fit to the signal region across the three event categories, in which the systematics are included as additional constrained degrees
of freedom. Exclusion limits are set for a number of DM models as calculated with  the CLs method~\cite{cls} using a profile likelihood ratio as the test-statistic in which systematic uncertainties are modelled as nuisance parameters. For each signal hypothesis tested, upper limits are placed on the ratio of the signal cross-section to the predicted cross-section, denoted as $\mu=\sigma/\sigma_{TH}$. Limits are calculated in terms of exclusion of regions in the $m_{\mathrm{MED}}-m_{\textrm{DM}}$ plane, assuming the four different mediators, 
by determining the points for which $\mu\ge1$ is excluded at 90\% CL or more.

Agreement between the expected SM backgrounds and data is observed at the percent level across the three categories. The largest single-bin local significance across the three categories, is 1.9$\sigma$ and corresponds to the excess seen in the last \ETm bin of the monojet category.

To compare direct detection experiments with collider experiments, we assume that the direct detection bounds can be interpreted under the Lagrangians given in Equation~\ref{eqg:lagrangians}. 
For (pseudo-)scalar models, we take the minimal assumption that only heavy quark (top and bottom) interactions are present. Such a choice limits the sensitivity for direct 
detection, however it allows for direct comparison between collider and direct detection without an additional assumptions on the light quark couplings~\cite{newpaper}. 
Under these assumptions, the resulting direct detection cross sections have been thoroughly discussed in the literature and the standard approaches have been followed~\cite{atruckload}. 

Figure~\ref{XXX}, shows the results compared with direct detection in the 2D-mediator vs dark matter mass plane. Additionally, in figure~\ref{XXX}, the results are also compared to constraints obtained from the observed cosmological relic density of DM as determined from  measurements of the cosmic microwave background by the WMAP and Planck experiments~\cite{Bennett:2003ba,Planck:2006aa} . In figure~\ref{YYY}, exclusion results are translated to the direct detection spin-dependent and spin-independent plot assuming dark matter coupling $g_{DM}=1$ and a vector,axial-vector, and scalar mediator . For vector mediators, the exclusion is strongest for direct detection for regions above 7~\GeV. Below 7~\GeV, the collider bound dominates. For axial mediators, the collider bound dominate for low dark matter masses and large medidator masses. For scalar bounds, the results are complementary to the direct detection bounds with coverage coming from the collider constraints at low dark matter mass. 

For pseudo-scalar interactions, direct detection bounds are strongly velocity suppressed. The most appropriate comparison is therefore to the most sensitive 
bounds from indirect detection from FermiLAT. These limits apply to the scenario in which dark matter is annihilated in the center of a galaxy producing a $\gamma$ ray signature. 
Recent excesses in this production have led to speculations that dark matter annihilation is mediated by a light pseudo-scalar~\cite{anothertuckload}. 
The production mechanism for these $\gamma$ rays can also be interpreted under the above Lagrangians for dark matter annihilation to b-quarks allowing for direct 
comparison with limits from this analysis~\cite{philnote,buckley,ob}. 

Figures~\ref{XXX} and ~\ref{YYY}, show the results of the collider bounds on a pseudo-scalar compared with indirect detection and relic density bounds. The results of the pseudoscalar extend well beyond the current bounds from FermiLAT and exclude an observed excess interpreted under the annihalation to b-quarks, \tau-leptons, and light quarks.   
