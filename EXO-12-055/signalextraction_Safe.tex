\section{Signal Extraction}
The presence of DM production will be observable as an excess of events with respect to the expectations for the 
SM backgrounds in a region at high missing transverse energy. Significant improvements in terms of sensitivity can be expected if 
several bins in $\ETm$ are considered simultaneously. 
A binned likelihood fit is performed in the range 250 \gev to 1000 \gev (or 200 \gev to 1000 \gev 
for the monojet category), where the binning is chosen to ensure each corresponding bin of the control regions  
is populated. The width of the highest $\ETm$ bin allows for ease of comparison to the previous CMS search~\cite{monojet1}. 

%\subsection{W+jets and Z+jets Backgrounds}\label{sec:zjetsmodel}
The accuracy with which the distribution for the major backgrounds (V+jets) is estimated is an essential part of this analysis. 
Data from three control regions are utilised in order to determine both the shape and normalization for the V+jets backgrounds in 
the signal region. The control regions are dimuon and photon control regions used to determine the \Zvvjets~contribution and a single-muon 
control region, which is used to determine the contribution from \Wlvjets. The events in the control regions are divided into the three 
categories, using the same selection criteria as described in Section~\ref{sec:selection}, with exceptions which are described in the following. 
%The procedure adopted is to derive a correction in bins of \ETm, using the data control regions, 
%which can be used to re-weight the simulated V+jets events in the signal region.  

The dimuon control region is defined using the same selection as for the signal region, 
but removing the muon veto. Instead, exactly two muons with opposite charges are 
required. The leading \pt muon needs to be isolated and pass a tight identification and 
the mass of the dimuon system, $m_{\mu\mu}$,  must be compatible with the Z boson mass, 
$60<m_{\mu\mu}<120$~\gev. The efficiency of the muon selection is measured 
in data and simulation using a tag-and-probe method. A scale-factor of $0.985\pm0.010$ is derived 
and applied twice (once per muon) to the simulated \Zjets~events passing the dimuon selection.

As the decay branching ratio of \Zmm~is approximately 6 times smaller than that to neutrinos, the resulting statistical 
uncertainty on the \Zvvjets~template becomes a dominant systematic uncertainty at large values of \ETm.
A complementary approach is to additionally use events in data that have a high-$p_{T}$ 
photon recoiling against jets to further 
constrain the \Zvvjets~background template~\cite{CMS-PAS-SUS-08-002}. At large boson transverse momentum, the kinematics 
of the \phojets~and \Zjets~processes become similar as the finite mass of the 
Z boson becomes less relevant. The photon control region consists of events which are selected using a trigger requiring an 
isolated  photon with transverse momentum greater than 150 GeV. 
The selected events are required to have at least one photon with $p_{T} > 170~\textrm{GeV}$ and 
$|\eta|<2.5$, ignoring photons in the ECAL transition region, $1.44 < |\eta|< 1.56$. 
Additionally, the photons are required to pass a set of quality and isolation cuts designed 
to reduce the contribution from jets faking photons. A conversion safe electron veto is 
applied to remove Drell-Yan events. The efficiency of the selection is  
measured in data using a tag-and-probe method in $Z\rightarrow ee$ events and compared to 
that measured in simulation~\cite{photon_purity}. A scale-factor of $0.97\pm0.01$ is applied to simulated events passing the 
selection to account for the difference in the photon identification efficiency between data and simulation. 

The single-muon control region is defined by selecting events 
with exactly one muon with \pt larger than 20\gev. The transverse 
mass of the event, defined  as  $m_{T}=\sqrt{2\ETm p_{T}^{\mu} (1-\cos \phi)}$, 
where $\phi$ is the angle between the muon and \ETm vector in the transverse plane, must also satisfy $50<m_{T}<100$ \gev, 
selecting $W\rightarrow\mu\nu$ decays. As in the dimuon control region, the muon efficiency scale-factor is 
applied to simulated events passing the selection. 

In the dimuon, single-muon and photon control regions, the dimuon pair, single-muon or the photon's 
momentum are removed and the \ETm ~is recalculated yielding a distribution of fake \ETm. The difference in the distribution of fake \ETm 
between the observed data and that expected from simulation in the control regions is used to derive a correction 
for the Z+jets and W+jets backgrounds in the signal region.
 
The \ETm spectra of the V+jets backgrounds is determined through the use of a likelihood fit, simultaneously across all bins 
in the three control regions. The expected number of events $N_{i}$ in a given bin $i$ of fake \ETm, for a particular event category, is given by, 
$N^{Z_{\mu\mu}|\gamma }_{i}=  \dfrac{{\mu^{\Zvv}_{i}}}{R^{Z|\gamma}_{i}}$, 
for the dimuon and photon control regions and  $N^{W}_{i} =  \dfrac{{\mu^{\Wlv}_{i}}}{R^{W}_{i}}$,
for the single-muon control region. The parameters $\boldsymbol{\mu}^{\Zvv}$ and $\boldsymbol{\mu}^{\Wlv}$ are the free parameters 
of the likelihood fit representing the yields of $\Zvvjets$ and $\Wlvjets$ in each bin of the signal region.

The transfer factors $R^{Z}_{i}$ account for the ratio of $BR(\Zvv)/BR(\Zmm)$ and 
the muon efficiency times acceptance in the dimuon control region, while $R^{\gamma}_{i}$ account for the ratio of differential 
cross-sections between the Z+jet and photon+jet processes and the efficiency times acceptance of the photon selection for the photon 
plus jet control region. These transfer factors are derived as the ratio of the number of \Zvvjets ~events in the signal region 
to that of \phojets~or \Zmmjets~events in the photon and dimuon control regions, from the simulation.
The differential boson $p_{T}$ cross-sections of photon and Z production are first corrected using NLO k-factors derived by  
comparing their $p_{T}$ distributions in events generated at single parton level with Madgraph5\_aMC@NLO~\cite{amcatnlo}, and subsequently showered,  
to the distributions produced at leading order before deriving the transfer factors. 

For the single-muon control region, the transfer factors $R^{W}_{i}$ account for the difference between the 
W+jets in the control region and the signal region due to the acceptance of the muon selection and lepton veto.
These are derived from the expectation from simulated W+jets events in the signal region $\Wmn$ and in the control region.

The likelihood is defined for each event category as the product over Poisson likelihoods for each bin in fake \ETm, in each of the three control regions, as

\begin{align*}\label{eqn:cancdlh}
\mathcal{L}_{\textrm{c}}(\boldsymbol{\mu}^{\textrm{c},\Zvv},\boldsymbol{\mu}^{\textrm{c},\Wlv},\boldsymbol{\theta},\boldsymbol{\phi}) &=        
                \prod_{i} \mathrm{Poisson}\left(d^{\textrm{c},\gamma}_{i} |B^{\textrm{c},\gamma}_{i}(\boldsymbol{\phi}) +\frac{ \mu^{\textrm{c},\Zvv}_{i} }{R^{\textrm{c},\gamma}_{i}(\boldsymbol{\theta})}   \right) \\
       &~\times \prod_{i} \mathrm{Poisson}\left(d^{\textrm{c},Z}_{i}      |B^{\textrm{c},Z}_{i}(\boldsymbol{\phi})      +\frac{ \mu^{\textrm{c},\Zvv}_{i} }{R^{\textrm{c},Z}_{i}     (\boldsymbol{\theta})}       \right ) \\
       &~\times \prod_{i} \mathrm{Poisson}\left(d^{\textrm{c},W}_{i}      |B^{\textrm{c},W}_{i}(\boldsymbol{\phi})      +\frac{ \mu^{\textrm{c},\Wlv}_{i} }{R^{\textrm{c},W}_{i}     (\boldsymbol{\theta})}       \right), \numberthis\label{eqn:candclh}
\end{align*}


where $d^{c,\gamma/Z/W}_{i}$ are the observed number of events in each bin of the photon, dimuon and single-muon control regions.
The superscript ``c'' has been introduced to signify the components which are independent in each of the categories. 
The expected contributions from background processes in the dimuon, single-muon and photon control regions are denoted $B^{Z}$, $B^{W}$ and 
$B^{\gamma}$ respectively.

Systematic uncertainties are modelled as constrained nuisance parameters, $\boldsymbol{\theta}$, which allow for variation of 
the transfer factors, $R^{\gamma/Z/W}$, in the fit and are treated as fully correlated between event categories.
These include theoretical uncertainties on the photon to Z differential cross-section ratio from renormalization and factorization scale uncertainties. 
Electroweak corrections to the ratio are not accounted for in the simulation. Such corrections 
have been derived independently 
and applied to the differential cross-section ratio. The full correction is also taken as an uncertainty on the ratio, 
which is of the order of 15\% when the boson (Z or $\gamma$) \pt is at the TeV scale~\cite{Kuhn:2005gv}. This uncertainty is assumed to be uncorrelated 
across bins of \ETm in order to be conservative. 
Additionally, the uncertainty in the muon and photon selection 
efficiency scale-factors are included and fully correlated across the event categories. The uncertainty on the 
muon efficiency scale-factor is correlated between the single-muon and dimuon control regions.

$B^{\gamma}$ is determined using a measurement of the purity of photon selection, as detailed in~\citep{photon_purity}. 
For photons with $\pt> 170$~\gev, the purity is $0.97\pm0.01$. 
Simulation is used to determine $B^{Z}$ and $B^{W}$, which are mostly from 
diboson production. A systematic uncertainty of 10\% is included for the background normalization~\cite{tagkey2015250} and correlated between the 
single-muon and dimuon control regions. A larger systematic uncertainty was considered to cover electroweak corrections at higher \ETm but this was found 
not to impact the results. As with the uncertainties on the transfer-factors, the uncertainties on the backgrounds 
are incorporated as nuisance parameters $\boldsymbol{\phi}$.
The full likelihood is then given by the product of $\mathcal{L}_{\textrm{c}}$ over the three event categories with a Gaussian constraint term for each of the nuisance parameters.

The likelihood is maximised (fit) with respect to the free parameters $\boldsymbol{\mu}^{c,\Zvv},~\boldsymbol{\mu}^{c,\Wlv}$ 
and the nuisance parameters $\boldsymbol{\theta}$ and $\boldsymbol{\phi}$, simultaneously yielding the predictions for the $\Zvvjets$ and $\Wlvjets$ backgrounds.
These re-weighted events are then binned to provide $\ETm$ templates for the two major backgrounds.
The nuisance parameter uncertainties, as well as the statistical uncertainties from the control regions, are propagated as shape and normalization 
variations for the Z+jets and W+jets $\ETm$ templates.  

The remaining backgrounds are expected to be much smaller than those from V+jets and are estimated directly from the simulation. 
Shape and normalization systematic uncertainties from the recoil corrections applied to these 
backgrounds are included to account for the uncertainty in the energy scale and resolution 
of jets. Additionally, a systematic uncertainty of 4\% is included for the top backgrounds due to the uncertainty of 
the b-tagging efficiency for the b-jet veto in the resolved category. Systematic uncertainties of 7\%, 10\% and 50\% are included for the top, diboson and QCD backgrounds respectively to account 
for the uncertainty in their production cross-sections. Finally, a systematic uncertainty of 2.6\% in the luminosity measurement~\cite{lumi} is included for 
all of the MC derived backgrounds.


%Table~\ref{tab:systematics} gives a summary of the systematic uncertainties on the backgrounds estimation which are propagated to the calculation of the limits.
%Experimental uncertainties in the signal resulting from jet energy scale and resolution and V-tagging efficiency are included in the signal model for all of the interpretations described.

%\begin{table*}[htbp]
%  \begin{center}
%    \topcaption{Systematic uncertainties and their relative effect on the expectation for the SM backgrounds. 
%      \label{tab:systematics}}
%    %\small
%    \begin{tabular}{llccc}
%      \hline
%      \hline
%      Systematic Uncertainty 	   	            & Process & Boosted & Resolved & Monojet  \\
%      \hline
%      \hline
%      Control region fits$^{\dagger}$ & $\Zvvjets$  &6-20\%  &7.6-44\%  &2.5-9.5\%  \\
%      			   	      & $\Wlvjets$  &10.5-55\% &14.5-320\%  &3.6-17\%  \\
%      \hline
%      Tau-id efficiency		& $\Wlvjets$      & 3.6\% & 3.6\% & 3.6\% \\
%      \hline
%      V-tag efficiency$^{\ddagger}$ 		& Dibosons, Top & \multicolumn{2}{c}{10\%,6\% } &  \\ 
%      \hline
%      b-tag efficiency 		& Top & \multicolumn{3}{c}{4\% } \\ 
%      \hline
%      \ETm recoil 		& Dibosons      & 0.6\% & 2.8\% & 0.3\%  \\ 
%       				& Top    	& 1.1\% & 1.8\% & 1.3\%  \\ 
%       				& $\Zlljets$    & 5.8\% & 9.4\% & 0.7\%  \\ 
%      \hline
%      $t\bar{t}$ norm  		& Top 	      & 7\%  & 7\%  & 7\% \\ 
%      Dibosons norm 		& Dibosons    & 10\% & 10\% & 10\%\\ 
%      QCD norm		 	& QCD 	      & 50\% & 50\% & 50\%\\ 
%      \hline
%      Luminosity  	 	& All except V+jets  & 2.6\% & 2.6\% & 2.6\% \\ 
%      \hline 
%      \hline
%    \end{tabular}\\
%    \end{center}
%    \bottomcaption{\footnotesize{$^{\dagger}$ The relevant components of the fit uncertainties relating to theory and muon/photon identification scale-factors 
%	in the control regions, described in Section~\ref{sec:zjetsmodel}, are included here and correlated between event categories.
%	The numbers here indicate the range of the size of the uncertainties (the smallest to largest) in any \ETm bin due to these fits but should not be 
%	interpreted as the systematic uncertainty which is propagated to the signal extraction.
%	For the boosted category, the uncertainty of 320\% is simply the result of a small fitted yield in one of the bins of the single muon control region.\\
%	$^{\ddagger}$ Uncertainty modeled as migration between the V-tagged (boosted and resolved) and monojet categories.
%    }}
%\end{table*}

